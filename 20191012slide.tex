\documentclass[dvipdfmx,11pt,notheorems]{beamer}
%%%% 和文用 %%%%%
\usepackage{bxdpx-beamer}
\usepackage{pxjahyper}
\usepackage{minijs}%和文用
\renewcommand{\kanjifamilydefault}{\gtdefault}%和文用

%%%% スライドの見た目 %%%%%
\usetheme{Madrid}
\usefonttheme{professionalfonts}
\setbeamertemplate{frametitle}[default][center]
\setbeamertemplate{navigation symbols}{}
\setbeamercovered{transparent}%好みに応じてどうぞ)
\setbeamertemplate{blocks}[rounded]
\useinnertheme{circles}
%\setbeamertemplate{footline}[page number]
%\setbeamerfont{footline}{size=\normalsize,series=\bfseries}
%\setbeamercolor{footline}{fg=black,bg=black}
%%%%

%%%% 定義環境 %%%%%
\usepackage{amsmath,amssymb}
\usepackage{amsthm}
\theoremstyle{definition}
\newtheorem{theorem}{定理}
\newtheorem{definition}{定義}
\newtheorem{proposition}{命題}
\newtheorem{lemma}{補題}
\newtheorem{corollary}{系}
\newtheorem{conjecture}{予想}
\newtheorem*{remark}{Remark}
\renewcommand{\proofname}{}
%%%%%%%%%

%%%%% フォント基本設定 %%%%%
\usepackage[T1]{fontenc}%8bit フォント
\usepackage{textcomp}%欧文フォントの追加
\usepackage[utf8]{inputenc}%文字コードをUTF-8
\usepackage{otf}%otfパッケージ
\usepackage{lxfonts}%数式・英文ローマン体を Lxfont にする
\usepackage{bm}%数式太字
%%%%%%%%%%

%%%%% PythonTeX %%%%%
\usepackage[makestderr]{pythontex}
\restartpythontexsession{\thesection}
 
\title[Number Theory with Python]{Pythonで楽しむ初等整数論}
\author[Hayao]{Hayao Suzuki}
\institute[Hiroshima 2019]{PyCon mini Hiroshima 2019}
\date{October 12, 2019}

\begin{document}

\begin{frame}[plain]\frametitle{}
\titlepage %表紙
\end{frame}

\begin{frame}\frametitle{Contents}
\tableofcontents %目次
\end{frame}

\section{自己紹介}

\begin{frame}\frametitle{自己紹介}

\begin{block}{お前誰よ}
\begin{description}
\item[名前] Hayao Suzuki(鈴木 駿)
\item[Twitter] @CardinalXaro
\item[ブログ] \url{https://xaro.hatenablog.jp/}
\item[専門] 数学 (組合せ論・グラフ理論)
\item[学位] 修士(工学)、電気通信大学
\item[仕事] 株式会社アイリッジでPythonプログラマをしている
\end{description}
\end{block}

\end{frame}

\begin{frame}\frametitle{自己紹介}

\begin{block}{技術書の査読}
\begin{itemize}
\item 『Effective Python』(オライリージャパン)など
\item \url{https://xaro.hatenablog.jp/}に一覧あります。
\end{itemize}
\end{block}

\begin{block}{いろんな発表}
\begin{itemize}
\item 「SymPyによる数式処理」(PyCon JP 2018)など
\item \url{https://xaro.hatenablog.jp/}に一覧あります。
\end{itemize}
\end{block}

\end{frame}

\section{初等整数論とは}

\begin{frame}\frametitle{初等整数論とは}

\begin{block}{整数論とは}
数の性質に関する数学の一分野
\end{block}

\begin{block}{初等整数論とは}
代数的な手法や解析的な手法を使わない数論
\end{block}

\begin{alertblock}{初等だから簡単ですね?よかった!}
初等とは予備知識がいらないという意味であり決して簡単ではありません。
\end{alertblock}

\end{frame}

\begin{frame}\frametitle{今日の発表}

\begin{block}{なぜPythonを使うのか}
\begin{itemize}
\item 比較的素直にプログラミングできる
\item 標準ライブラリも外部ライブラリも豊富にある
\item Guido van Rossum氏はアムステルダム大学で数学と計算機科学を専攻していた
\end{itemize}
\end{block}

\begin{block}{今回使うもの}
\begin{itemize}
\item Python 3.7.x
\item SymPy(記号計算ライブラリ、数論ライブラリも豊富)
\item Silverman『はじめての数論 原著第3版』(丸善出版)(ネタ元)
\end{itemize}
\end{block}

\end{frame}

\begin{frame}\frametitle{今日の発表}

\begin{block}{Pythonで楽しむ初等整数論}
\begin{itemize}
\item 定理の結果をPythonでプログラミングしてみよう
\item 定理の証明からPythonでプログラミングしてみよう
\end{itemize}
\end{block}

\begin{exampleblock}{でも...難しいんでしょう?}
割り算ができれば大丈夫!(個人差があります)
\end{exampleblock}

\end{frame}

\section{ピタゴラス数}

\begin{frame}\frametitle{Pythonで楽しむ初等整数論}
\huge{定理の結果をPythonでプログラミングしてみよう}
\end{frame}

\begin{frame}\frametitle{ピタゴラスの定理}

\begin{theorem}[Pythagoras]
直角三角形の斜辺の長さを$c$とし、それ以外の辺の長さを$a, b$とする。
そのとき、
\begin{equation*}
a^{2} + b^{2} = c^{2}
\end{equation*}
 が成り立つ。
\end{theorem}

\end{frame}

\begin{frame}\frametitle{ピタゴラス数}

\begin{definition}[ピタゴラス数]
$a^{2} + b^{2} = c^{2}$が成り立つ自然数の組$(a, b, c)$をピタゴラス数と呼ぶ。
\end{definition}

\end{frame}

\begin{frame}[fragile]\frametitle{ピタゴラス数を計算しよう:根性編}

\begin{block}{根性で計算だ!}
\begin{pyverbatim}
from itertools import product

for a, b, c in product(range(1, 20), repeat=3):
    if a ** 2 + b ** 2 == c ** 2:
        print(a, b, c)
\end{pyverbatim}
\end{block}

\end{frame}

\begin{frame}[fragile]\frametitle{ピタゴラス数を計算しよう:根性編}

\begin{block}{根性で計算したぞ!}
\begin{pycode}
from itertools import product

for a, b, c in product(range(1, 20), repeat=3):
    if a ** 2 + b ** 2 == c ** 2:
        print(a, b, c)
        print("\n")
\end{pycode}
\end{block}

\end{frame}

\begin{frame}\frametitle{既約ピタゴラス数}

\begin{proposition}[ピタゴラス数からピタゴラス数を作る]
自然数の組$(a, b, c)$がピタゴラス数ならば、
任意の自然数$n$に対して$(na, nb, nc)$もまたピタゴラス数である。
\end{proposition}

\begin{proof}[証明]
\begin{equation*}
(na)^{2} + (nb)^{2} = n^{2} (a^{2} + b^{2}) = (nc)^{2}.
\end{equation*}
\end{proof}

\begin{definition}[既約ピタゴラス数]
互いに素であるピタゴラス数を既約ピタゴラス数と呼ぶ。
\end{definition}

\end{frame}

\begin{frame}[fragile]\frametitle{既約ピタゴラス数を計算しよう}

\begin{block}{ちょっと工夫した}
\begin{pyverbatim}
from itertools import combinations
from math import gcd

for a, b, c in combinations(range(1, 50), 3):
    if a ** 2 + b ** 2 == c ** 2 and gcd(a, b) == 1:
        print(a, b, c)
\end{pyverbatim}
\end{block}

\begin{block}{ちょっと工夫した}
\begin{itemize}
\item $a,b,c$はそれぞれ違う数である
\item $a,b$は互いに素である
\end{itemize}
\end{block}

\end{frame}

\begin{frame}[fragile]\frametitle{既約ピタゴラス数を計算しよう}

\begin{block}{既約ピタゴラス数の計算}
\begin{pycode}
from itertools import combinations
from math import gcd

for a, b, c in combinations(range(1, 50), 3):
    if a ** 2 + b ** 2 == c ** 2 and gcd(a, b) == 1:
        print(a, b, c)
        print("\n")
\end{pycode}
\end{block}

\end{frame}


\begin{frame}\frametitle{既約ピタゴラス数をもっと上手く計算しよう}

\begin{lemma}
既約ピタゴラス数$(a, b, c)$の$a, b$の一方は偶数でもう一方は奇数である。
また、$c$は常に奇数である。
\end{lemma}

\begin{proof}[証明]
自然数の組$(a, b, c)$を既約ピタゴラス数とする。

まず、$a, b$が共に偶数ならば、ピタゴラス数の定義から$c$も偶数となり、$(a, b, c)$は共通因数2を持つことになり仮定に反する。
次に、$a, b$が共に奇数ならば、ピタゴラス数の定義から$c$は偶数となる。
このとき、$a = 2x + 1, b = 2y + 1, c = 2z$とすると
\begin{equation*}
a^{2} + b^{2} = c^{2} \Rightarrow 2(x^{2} + x + y^{2} + y) + 1 = 2z^{2}
\end{equation*}
となり矛盾。

よって、$a, b$の内、一方は偶数でもう一方は奇数である。
また、ピタゴラス数の定義から$c$は奇数となる。
\end{proof}

\end{frame}

\begin{frame}\frametitle{既約ピタゴラス数をもっと上手く計算しよう}
\begin{lemma}
既約ピタゴラス数$(a, b, c)$において、$a$を奇数、$b$を偶数とする。
このとき、$(c-b)(c+b)$は互いに素である。
\end{lemma}

\begin{proof}[証明]
自然数の組$(a, b, c)$を既約ピタゴラス数とし、$a$を奇数、$b$を偶数とする。
仮定から、$a^{2} = c^{2} -b^{2} = (c-b)(c+b)$とできる。
$c-b, c+b$の共通因数を$d$とすると、$d$は$(c+b)+(c-b)=2c, (c+b)-(c-b)=2b$も割り切る。
仮定から$b, c$は既約なので$d$は1または2である。
しかし、$(c-b)(c+b)=a^{2}$であり、$a$は奇数なので$d$は1に他ならない。

\end{proof}

\end{frame}

\begin{frame}\frametitle{既約ピタゴラス数をもっと上手く計算しよう}
\begin{lemma}
既約ピタゴラス数$(a, b, c)$において、$a$を奇数、$b$を偶数とする。
このとき、$c-b, c+b$はそれぞれ平方数となる。
\end{lemma}

\begin{proof}[証明]
自然数の組$(a, b, c)$を既約ピタゴラス数とし、$a$を奇数、$b$を偶数とする。
仮定から、$a^{2} = c^{2} -b^{2} = (c-b)(c+b)$とできる。
$c-b, c+b$は互いに素であり、かつ$(c-b)(c+b)=a^{2}$なので、$(c-b)(c+b)$は平方数となる。
よって、$c-b, c+b$がそれぞれ平方数となる。
\end{proof}

\end{frame}


\begin{frame}\frametitle{既約ピタゴラス数をもっと上手く計算しよう}

\begin{proposition}
互いに素な奇数$s, t (s > t)$対して、 $a = st, b = \displaystyle \frac{s^{2}-t^{2}}{2}, c = \displaystyle \frac{s^{2}+t^{2}}{2}$は既約ピタゴラス数となる。
\end{proposition}

\begin{proof}[証明]
自然数の組$(a, b, c)$を既約ピタゴラス数とし、$a$を奇数、$b$を偶数とする。
仮定から、$a^{2} = c^{2} -b^{2} = (c-b)(c+b)$とできる。
$c-b, c+b$はそれぞれ平方数なので、$c+b=s^{2}, c-b=t^{2}$とできる。
よって、
\begin{equation*}
c = \frac{s^{2} + t^{2}}{2}, b = \frac{s^{2} - t^{2}}{2}
\end{equation*}
となる。
また、
\begin{equation*}
a = \sqrt{(c-b)(c+b)} = st
\end{equation*}
となる。
\end{proof}

\end{frame}

%\begin{frame}\frametitle{お待たせしました}
%\huge{やることが...やることが多い...!!}
%\end{frame}


\begin{frame}[fragile]\frametitle{既約ピタゴラス数をもっと上手く計算しよう}

\begin{block}{上手く工夫した}
\begin{pyverbatim}
from itertools import combinations
from math import gcd

for t, s in filter(
    lambda x: gcd(*x) == 1,
    combinations(range(1, 10, 2), r=2),
):
    print(
        s * t,
        int((s ** 2 - t ** 2) / 2),
        int((s ** 2 + t ** 2) / 2),
    )
\end{pyverbatim}
\end{block}

\end{frame}

\begin{frame}[fragile]\frametitle{既約ピタゴラス数をもっと上手く計算しよう}

\begin{block}{既約ピタゴラス数をもっと上手く計算した}
\begin{pycode}
from itertools import combinations
from math import gcd

for t, s in filter(
    lambda x: gcd(*x) == 1,
    combinations(range(1, 10, 2), r=2),
):
    print(
        s * t,
        int((s ** 2 - t ** 2) / 2),
        int((s ** 2 + t ** 2) / 2),
    )
    print("\n")
\end{pycode}
\end{block}

\end{frame}

\begin{frame}\frametitle{Pythonで楽しむ初等整数論}

\begin{block}{定理の結果をPythonでプログラミングしてみよう}
\begin{itemize}
\item 結果をそのまま実装して具体例を観察する
\item 意外な法則が見つかる(かも)
\end{itemize}
\end{block}

\end{frame}

\section{2つの平方数の和で表せる素数}

\begin{frame}\frametitle{ピタゴラス数(復習)}

\begin{definition}[ピタゴラス数]
$a^{2} + b^{2} = c^{2}$が成り立つ自然数の組$(a, b, c)$をピタゴラス数と呼ぶ。
\end{definition}

\begin{definition}[既約ピタゴラス数]
互いに素であるピタゴラス数を既約ピタゴラス数と呼ぶ。
\end{definition}

\end{frame}

\begin{frame}[fragile]\frametitle{既約ピタゴラス数の斜辺の性質}

\begin{block}{既約ピタゴラス数の斜辺を4で割ると...?}
\begin{pyverbatim}
from itertools import combinations
from math import gcd

for t, s in filter(
    lambda x: gcd(*x) == 1,
    combinations(range(1, 10, 2), r=2),
):
    c = int((s ** 2 + t ** 2) / 2)
    print(f"{c} を4で割った余りは{c % 4}")
\end{pyverbatim}
\end{block}

\end{frame}

\begin{frame}[fragile]\frametitle{既約ピタゴラス数の斜辺の性質}

\begin{block}{既約ピタゴラス数の斜辺を4で割ると...?}
\begin{pycode}
from itertools import combinations
from math import gcd

for t, s in filter(
    lambda x: gcd(*x) == 1,
    combinations(range(1, 10, 2), r=2),
):
    c = int((s ** 2 + t ** 2) / 2)
    print(f"{c} を4で割った余りは{c % 4}")
    print()
\end{pycode}
\end{block}

\end{frame}


\begin{frame}\frametitle{2つの平方数の和で表せる素数}

\begin{theorem}[Fermatの2平方和定理]
奇素数$p$が2つの平方数の和で表せることの必要十分条件は$p \equiv 1 \pmod{4}$である。
\end{theorem}

\begin{proof}[証明$(\Rightarrow)$]
奇素数$p$が2つの平方数の和で表せるならば、$p=x^{2}+y^{2}$となる。
$p$は奇数なので、$x$を偶数、$y$を奇数としてよい。
このとき、$x=2m, y=2n+1$とすると
\begin{equation*}
p = x^{2}+y^{2} = (2m)^{2} + (2n + 1)^{2} = 4(m^{2} + n^{2} + n) + 1
\end{equation*}
より、$p$は4を法として1と合同である。

\end{proof}

\end{frame}

\begin{frame}\frametitle{証明のスケッチ}

\begin{proof}[証明$(\Leftarrow)$のスケッチ]

\begin{itemize}
\item $p$を4を法として1と合同な素数とする。
\item いい感じに$A^{2}+B^{2} = Mp$なる$(A, B, M)$を探す。
\item $(A, B, M)$からいい感じに$a^{2}+b^{2}=mp, m \leq M-1$なる$(a, b, m)$を探す。
\item $M$が1になるまで繰り返す。
\end{itemize}

\end{proof}

\end{frame}

\begin{frame}\frametitle{証明$(\Leftarrow)$のスケッチ}

\begin{lemma}
\begin{equation*}
(u^{2} + v^{2})(A^{2}+B^{2}) = (uA+vB)^{2} + (vA-uB)^{2}
\end{equation*}
\end{lemma}

\begin{proof}[証明]

\begin{equation*} 
\begin{split}
&(uA+vB)^{2} + (vA-uB)^{2}  \\
&= (u^{2}A^{2} + 2uAvB + v^{2}B^{2}) + (v^{2}A^{2}-2vAuB + u^{2}B^{2})  \\
 &= u^{2}A^{2}+v^{2}B^{2} + v^{2}A^{2}+u^{2}B^{2} \\
 &= (u^{2} + v^{2})(A^{2}+B^{2}).
 \end{split}
\end{equation*}

\end{proof}

\end{frame}

\begin{frame}\frametitle{証明$(\Leftarrow)$のスケッチ}

\begin{lemma}
合同方程式$x^{2} \equiv -1 \pmod{p}$は解を持つ。
\end{lemma}

\begin{proof}[証明(天下り)]
平方剰余の相互法則から導ける。
\end{proof}

\begin{exampleblock}{いい感じに$A^{2}+B^{2} = Mp$なる$(A, B, M)$を探す}
合同方程式$x^{2} \equiv -1 \pmod{p}$の解を$A$とし、$B=1$とすると、$A^{2}+B^{2}$は$p$の倍数である。
また、$\displaystyle M=\frac{A^{2} + B^{2}}{p}$となる。
\end{exampleblock}

\end{frame}

\begin{frame}[fragile]\frametitle{いい感じに$A^{2}+B^{2} = Mp$なる$(A, B, M)$を探す}

\begin{block}{合同方程式$x^{2} \equiv -1 \pmod{p}$の解を算出する(天下り)}
\begin{pyverbatim}
from random import randint
from sympy.ntheory import isprime, legendre_symbol


def solve_quadratic(p):
    if not (isprime(p) and p % 4 == 1):
        raise ValueError
    while True:
        a = randint(1, p - 1)
        b = pow(a, (p - 1) // 4, p)
        if legendre_symbol(a, p) == -1 and 1 <= b < p:
            return b
\end{pyverbatim}
\end{block}

\end{frame}

\begin{frame}[fragile]\frametitle{いい感じに$A^{2}+B^{2} = Mp$なる$(A, B, M)$を探す}

\begin{block}{いい感じに$(A, B, M)$を探す}
\begin{pyverbatim}
def sums_of_two_squares(p):
    """4を法として1と合同な素数を2つの平方数の和で表す"""

    if not (isprime(p) and p % 4 == 1):
        raise ValueError
            
    A, B = solve_quadratic(p), 1
    M = divmod(A ** 2 + B ** 2, p)[0]
\end{pyverbatim}
\end{block}

\end{frame}

\begin{frame}[fragile]\frametitle{いい感じに$a^{2}+b^{2}=mp$を探す}

\begin{block}{いい感じに$a^{2}+b^{2}=Mr$を探す}
以下の条件を満たすように$a, b$を選ぶ。
\begin{align*} 
a &\equiv A \pmod{M} \\ 
b &\equiv B \pmod{M} \\ 
-\frac{1}{2}M &\leq a, b \leq \frac{1}{2}M
\end{align*}
\end{block}

\begin{exampleblock}{$r$と$M$の関係の秘密}
\begin{equation*}
r = \frac{a^{2} + b^{2}}{M} \leq \frac{\left (\frac{M}{2} \right)^{2} + \left (\frac{M}{2} \right)^{2}}{M} = \frac{M}{2} < M.
\end{equation*}
\end{exampleblock}

\end{frame}

\begin{frame}[fragile]\frametitle{いい感じに$a^{2}+b^{2}=Mr$を探す}

\begin{block}{いい感じに$a^{2}+b^{2}=Mr$を探す}
\begin{pyverbatim}
from sympy.ntheory.modular import solve_congruence


def sums_of_two_squares(p):
    """4を法として1と合同な素数を2つの平方数の和で表す"""

    if not (isprime(p) and p % 4 == 1):
        raise ValueError
                    
    A, B = solve_quadratic(p), 1
    M = divmod(A ** 2 + B ** 2, p)[0]

    a = solve_congruence((A, M), symmetric=True)[0]
    b = solve_congruence((B, M), symmetric=True)[0]
    r = divmod(a ** 2 + b ** 2, M)[0]
\end{pyverbatim}
\end{block}

\end{frame}

\begin{frame}[fragile]\frametitle{総仕上げ}
\begin{block}{2つの方程式}
\begin{align*}
A^{2} + B^{2} &= Mp \\
a^{2} + b^{2} &= Mr
\end{align*}
\end{block}

\begin{block}{2つの方程式の関係}
\begin{equation*} 
\begin{split}
            & (a^{2} + b^{2})(A^{2}+B^{2}) = M^{2}rp  \\
\Rightarrow & (aA + bB)^{2} + (bA - aB)^{2} = M^{2}rp \\
\Rightarrow & \left(\frac{aA + bB}{M}\right)^{2} + \left(\frac{bA - aB}{M}\right)^{2} = rp
 \end{split}
\end{equation*}
\end{block}

\end{frame}

\begin{frame}[fragile]\frametitle{2つの平方数の和で表せる素数}

\begin{block}{2つの平方数の和で表せる素数}
\begin{pyverbatim}
def sums_of_two_squares(p):
    A, B = solve_quadratic(p), 1
    M = divmod(A ** 2 + B ** 2, p)[0]
    while True:
        a = solve_congruence((A, M), symmetric=True)[0]
        b = solve_congruence((B, M), symmetric=True)[0]
        r = divmod(u ** 2 + v ** 2, M)[0]

        s = abs((a * A + b * B) // M)
        t = abs((b * A - a * B) // M)
        if r == 1:
            print(f"${s}^2 + {t}^2={p}$")
            return
        else:
            A, B, M = s, t, r
\end{pyverbatim}
\end{block}

\end{frame}

\begin{frame}[fragile]\frametitle{2つの平方数の和で表せる素数}

\begin{block}{2つの平方数の和で表せる素数}
\begin{pyverbatim}
from sympy import primerange

primes_1_mod_4 = (
    p for p in primerange(1, 100) if p % 4 == 1
)

for p in primes_1_mod_4:
    sums_of_two_squares(p)
\end{pyverbatim}
\end{block}

\end{frame}

\begin{frame}[fragile]\frametitle{2つの平方数の和で表せる素数}

\begin{block}{2つの平方数の和で表せる素数}
\begin{pycode}
from random import randint
from sympy import primerange
from sympy.ntheory import isprime, legendre_symbol
from sympy.ntheory.modular import solve_congruence


def solve_quadratic(p):
    """x^2 \equiv -1 (mod p)の解を計算する"""
    if not (isprime(p) and p % 4 == 1):
        raise ValueError(f"{p}は4を法として1と合同な素数である必要があります。")
    while True:
        a = randint(1, p - 1)
        b = pow(a, (p - 1) // 4, p)
        if legendre_symbol(a, p) == -1 and 1 <= b < p:
            return b


def sums_of_two_squares(p):
    """4を法として1と合同な素数を2つの平方数の和で表す"""

    if not (isprime(p) and p % 4 == 1):
        raise ValueError(f"{p}は4を法として1と合同な素数である必要があります。")

    A, B = solve_quadratic(p), 1
    M = divmod(A ** 2 + B ** 2, p)[0]
    if M == 1:
        print(f"${A}^2 + {B}^2={p}$")
        return

    while True:
        u = solve_congruence((A, M), symmetric=True)[0]
        v = solve_congruence((B, M), symmetric=True)[0]
        r = divmod(u ** 2 + v ** 2, M)[0]
        s = abs((u * A + v * B) // M)
        t = abs((v * A - u * B) // M)

        if r == 1:
            assert s ** 2 + t ** 2 == p
            print(f"${s}^2 + {t}^2={p}$")
            return
        else:
            A, B, M = s, t, r


primes_1_mod_4 = (
    p for p in primerange(1, 100) if p % 4 == 1
)

for p in primes_1_mod_4:
    sums_of_two_squares(p)
    print()
\end{pycode}
\end{block}

\end{frame}

\begin{frame}\frametitle{Pythonで楽しむ初等整数論}

\begin{block}{定理の証明からPythonでプログラミングしてみよう}
\begin{itemize}
\item 構成的な証明はプログラミングできる(こともある)
\item プログラミングできれば証明を理解できる(かもしれない)
\end{itemize}
\end{block}

\end{frame}

\section{まとめ}

\begin{frame}\frametitle{まとめ}

\begin{block}{なぜPythonを使うのか}
\begin{itemize}
\item 比較的素直にプログラミングできる
\item 標準ライブラリも外部ライブラリも豊富にある
\item 困ったらSymPyを当たれば解決することが多い
\end{itemize}
\end{block}

\begin{block}{定理の結果をPythonでプログラミングしてみよう}
\begin{itemize}
\item 結果をそのまま実装して具体例を観察する
\item 意外な法則が見つかる(かも)
\end{itemize}
\end{block}


\begin{block}{定理の証明からPythonでプログラミングしてみよう}
\begin{itemize}
\item 構成的な証明はプログラミングできる(こともある)
\item プログラミングできれば証明を理解できる(かもしれない)
\end{itemize}
\end{block}

\end{frame}

\end{document}
